\documentclass[conference]{IEEEtran}
\IEEEoverridecommandlockouts
% The preceding line is only needed to identify funding in the first footnote. If that is unneeded, please comment it out.
\usepackage{comment}
\usepackage{subfigure}
\usepackage{multirow}
\usepackage{url}
\usepackage{cite}
\usepackage{flushend}
\usepackage{amsmath,amssymb,amsfonts}
\usepackage{algorithmic}
\usepackage{graphicx}
\usepackage{textcomp}
\usepackage{xcolor}
\def\BibTeX{{\rm B\kern-.05em{\sc i\kern-.025em b}\kern-.08em
    T\kern-.1667em\lower.7ex\hbox{E}\kern-.125emX}}
\begin{document}

\title{A Multi-Representation Ensemble Approach to Classifying Vocal Diseases}
\author{\IEEEauthorblockN{Mingxuan Ju, Zhengkai Jiang, Yufan Chen and Soumya Ray}
\IEEEauthorblockA{Department of Electrical Engineering and Computer Science, Case Western Reserve University, Cleveland, OH 44106}
\IEEEauthorblockA{Email: {\{mxj255, zxj89, yxc775, sray\}@case.edu}}
}
\maketitle
\begin{abstract}
The goal of the IEEE 2018 FEMH Voice Data Challenge was to develop an effective algorithmic approach to classifying voice samples as normal or pathological, and further subdivide the pathological samples into three types. We adopted a multi-representation ensemble approach to the task. We designed a pipeline with three classification stages, where each stage used a combination of supervised, semi-supervised and multiple-instance learners. This approach was able to achieve a sensitivity of 89\% and specificity of 76\% in classifying normal from pathological samples and an unweighted average recall (UAR) of 60.67\% in subclassifying pathological samples into three types.
\end{abstract}
%\tableofcontents

\section{Introduction and Background}
The goal of the IEEE 2018 FEMH Voice Data Challenge was to develop an effective algorithmic approach to classifying voice samples as normal or pathological, and further subdivide the pathological samples into three types: vocal palsy, phonotrauma or neoplasm. As noted by the challenge organizers, voice disorders can severely affect quality of life, since humans primarily rely on speech for communication. At the same time, diagnoses of vocal disorders depend on larygeal endoscopy, which requires human expertise and specialized equipment. Thus there is a need for algorithms that could for example be used to triage cases to focus scarce resources.

The task of differentiating normal from pathological voice samples has been investigated in previous work, with a summary of previous approaches in~\cite{b9}. Among these approaches are artificial neural network with Mel-Frequency-Cepstral-Coefficients(MFCC), feature vectors from multiple frames that collectively represent the power spectrum of a sample~\cite{b5}. Others have used wavelet transform with support vector machines (SVMs)~\cite{b4}. Recently a deep neural network~\cite{b9} has been used by dividing the raw waveform into multiple segments and extracting normalized MFCC features from them.


%\subsection{Background:}
	Vocal classification is perceived as one of the challenging tasks in medical field. Many researchers did meaningful work on designing optimal classifiers which can help diagnose vocal diseases from patient's voice record. Some vocal diseases such as neoplasm are notoriously hard to distinguish solely by listening due to noise and subtlety of symptom. 
\subsection{Related work:}
	Vahid Majidnezhad tried artificial neural network with Mel-Frequency-Cepstral-Coefficients as feature vectors to achieve optimal result on vocal pathology classfication\cite{b5}. P. Kukharchik also used wavelet transform with support vector machine to optimize classifiers' performance\cite{b4}. However, there are not any researcher who has tried multiple-ensemble classification and divide mul-class classification into multiple binary classification in vocal pathology classification.
	  % Yufan Chen
\section{Methodology}
\subsection{Data Preprocessing:}
\begin{itemize}
	\item Silence and Noise Removal\\
	Doing a spot check on both training and testing set, we discovered that there exists silence or noise at the beginning of most audio files. We removed those silence or noise parts by calculating the average loudness of each audio file, and recursively comparing the value of 30\% of average loudness to first $\frac{3}{4}$ of the audio file.\\
	\item Equalizing Loudness:\\
	We also figured out that ,between training set and testing sets, there is a loudness difference not representative of class label. So, we linearly equalized average loudness of each file by the average loudness of all audio files. \\
\end{itemize}
\subsection{Feature Extraction:}
\begin{itemize}
	\item Zero Crossing Rate(ZCR):\\
	The rate of sign-changes of the signal during the duration of a particular frame.\cite{b1}\\
	\item Energy:\\
	The sum of squares of the signal values, normalized by the respective frame length.\cite{b2}\\
	\item Entropy of Energy(EE):\\
	The entropy of sub-frames' normalized energies. It can be interpreted as a measure of abrupt changes.\cite{b2}\\
	\item Spectral Centroid(SC):\\
	It indicates where the "center of mass" of the spectrum is located. Perceptually, it has a robust connection with the impression of "brightness" of a sound.\cite{b3}\\
	\item Spectral Entropy(SE):\cite{b1}\\
	Entropy of the normalized spectral energies for a set of sub-frames.\cite{b3}\\
	\item Spectral Spread(SS):\\
	The second central moment of the spectrum.\cite{b3}\\
	\item Spectral Entropy:\\
	Entropy of the normalized spectral energies for a set of sub-frames.\cite{b3}\\
	\item Spectral Flux:\\
	The squared difference between the normalized magnitudes of the spectra of the two successive frames.\cite{b3}\\
	\item Spectral Rolloff:\\
	The frequency below which 90\% of the magnitude distribution of the spectrum is concentrated.\cite{b3}\\
	\item Mel Frequency Cepstral Coefficients(MFCCs):\\
	Mel Frequency Cepstral Coefficients form a cepstral representation where the frequency bands are not linear but distributed according to the mel-scale.\cite{b4}\\
	\item Chroma Vector:\\
	A 12-element representation of the spectral energy where the bins represent the 12 equal-tempered pitch classes of western-type music (semitone spacing).\cite{b5}\\
	\item Chroma Deviation:\\
	The standard deviation of the 12 chroma coefficients.\cite{b5}\\
	\item Local Amplitude Minimum:\\
	The number of local minimum amplitude peaks.\\
\end{itemize}
First 13 features are widely used in the field of machine learning for audio classification. We utilize an existing library\cite{b6} that implements first those features. The last feature Local Amplitude Minimum is introduced by us to help classifiers distinguishing normal audio and abnormal audio, which turns out to be extremely efficient. \\
\subsection{Validation Method:}
Before mentioning detailed structure and algorithm we have attempted, I want to point out that we measure accuracy using stratified 10-fold cross validation(CV). Any classifier's accuracy is the weighted average accuracy of numbers we get from 10 folds.  \\
\indent However, for transductive learning algorithm such as Transductive Support Vector Machine or Label Propagation, we could not do cross validation since there is no validation set for transductive learning. Accuracy we have on those algorithms is the accuracy on whole training set, in other word, the rate of convergence. 
\subsection{Attempted Model:}
Tree-Based Algorithms:
\begin{itemize}
	\item Decision Tree:\\
	Utilizing tree structures, decision tree algorithm calculates entropy of each feature and splits node based on the rank of features with respect to their entropies. Probably due to the fact that different classes are representative in different features and skewed class distribution, a single decision tree could not solve this learning problem and we only get 62\% accuracy on CV.\\
	\item Random Forest:\\
	Instead of having a single tree structure, Random Forest fits a number of decision tree classifiers on various sub-samples of the dataset.\cite{b7} Using Random Forest classifier, we get 89\% accuracy on CV. But even with such high accuracy, Random Forest performs poorly on test set according to the feedback from oracle. We analyze that the differences on performance between training set and testing set probably result from skewed data distribution and over-fitting. \\
\end{itemize}
\indent Based on cross validation results on experiments we have done, it seems like tree-based algorithm generally is inappropriate for this learning task. \newline\\
Support Vector Machines(SVMs):\\
\begin{itemize}
	\item Linear SVM:\\
	Based on the features' distribution of each training example, SVM differentiates classes by building a hyperplane that maximizes the margin between classes. And for Linear SVM, the decision boundary is a linear function. After tuning loss penalty and class weight, we get 91\% accuracy on CV and a relatively reasonable classification on test set according to the feedback from oracle.\\
	\item SVM with Radial Basis Function kernel(SVM-RBF):\\
	Pretty similar to Linear SVM, SVM-RBF differs by the non-linear kernel function. After having a reasonable result from Linear SVM, we were excited on trying SVM-RBF(with tuned loss penalty and gamma). It ends up having 89\% on CV and similar test set performance. However in the late phase, we discarded SVM-RBF because with the number of features we have, it is possible that a non-linear kernel function over-fits. \\
\end{itemize}

\subsection{Model Selection:}
	Our whole model-building infrastructure is based on scikit-learn\cite{b7}\cite{b8}. \\
	We tried K-Nearest Neighbor, Support Vector Machine, Boosting, Random Forest, Extratrees, Multiple Instance Learning, Label Propagation. After few experiments, it seemed like tree algorithms performed poorly on this task and we finalized our focus on SVM, Label Propagation, SVC and MILR with pipeline. This classifier undercalled Normal and Neoplasm patients, which we think is resulted from different class distributions between training and testing set. So, we converted this relatively complex learning task into three less complicated tasks: Normal vs. Pathological, Vocal vs. Rest of Diseases, and Phonotrauma vs. Neoplasm.[Fig. 1.] The reason why we design the pipeline this way is based on the difficulty of classification. (From easiest Normal vs. Pathological to hardest Phonotrauma vs. Neoplasm)
	\begin{figure}[htbp]
		\begin{center}
			\includegraphics[scale=0.35]{Diagram_1.png}
		\end{center}
		\caption{Pipeline Ensemble}
	\end{figure}
	\subsection{Hyper Parameter Tuning:}
		Since the size of training set is getting smaller and smaller as we propagate through the pipeline, in order to get trustworthy accuracy to tune parameter for SVC, for all three ensembles, we use the same parameter tuned in the first ensemble. The process is pretty straightforward; we simply set up two loops, one of which for C and one of which for gamma. The tuned parameters are 10 for C, and 0.01 for gamma. Also due to the skewed distribution in training set, we modify the class weight with respect to the proportion of class. 

   % Mingxuan Ju
\section{Feature Extraction}
\begin{itemize}
	\item Zero Crossing Rate(ZCR):\\
	The rate of sign-changes of the signal during the duration of a particular frame.\cite{b1}We calculate ZCR by equation: $\frac{1}{T-1}\sum_{t = 1}^{T - 1}(s_t s_{t-1})$ where s is single with length T.\\
	\item Energy:\\
	The sum of squares of the signal values, normalized by the respective frame length.\cite{b2} We calculate EE by equation: Energy = $W_{potential} + W_{kinetic} = \int_{V}^{} \frac{p^2}{2p_0c^2} dV + \int_{V}^{} \frac{pv^2}{2}dV$.\\
	\item Entropy of Energy(EE):\\
	The entropy of sub-frames' normalized energies. It can be interpreted as a measure of abrupt changes.\cite{b2} Setting number of short blocks n to 10, we calculate EE by equation: EE = $\frac{(\frac{L}{n})^2}{Eol +eps} \log_{2}({(\frac{(\frac{L}{n})^2}{Eol +eps} + eps})$, where L is the length of audio frame, eps is the learning rate and Eol is the total energy of this frame.\\
	\item Spectral Centroid(SC):\\
	It indicates where the "center of mass" of the spectrum is located. Perceptually, it has a robust connection with the impression of "brightness" of a sound.\cite{b3} We calculate SC through equation: SC = $\frac{\sum_{n = 0}^{N - 1} f(n)x(x)}{\sum_{n = 0}^{N - 1} x(n)}$, where f(n) is the frequency of current bin and x(n) is weighted frequency value.\\
	\item Spectral Energy(SE):\cite{b1}\\
	Entropy of the normalized spectral energies for a set of sub-frames.\cite{b3}. We calculate SE through equation: SE = $\sum_{i = 1}^{N} \frac{1}{N} |X(w_i)|^2$ \\
	\item Spectral Spread(SS):\\
	The second central moment of the spectrum.\cite{b3}. With SC, we can calculate SS by equation: SC = $\sqrt{\frac{\sum_{k = 0}^{\frac{N}{2}} (f_k-SC)^2 |X(k)|^2}{\sum_{k = 0}^{\frac{N}{2}} |X(k)|^2 }}$\\
	\item Spectral Entropy:\\
	Entropy of the normalized spectral energies for a set of sub-frames.\cite{b3} We calculate Spectral Energy by equation : Spectral Entropy = $-\sum_{i = 1}^{n} \frac{\frac{1}{N} |X(w_i)|^2}{\sum_{i}^{}\frac{1}{N} |X(w_i)|^2} \ln\frac{\frac{1}{N} |X(w_i)|^2}{\sum_{i}^{}\frac{1}{N} |X(w_i)|^2}$.\\
	\item Spectral Flux:\\
	The squared difference between the normalized magnitudes of the spectra of the two successive frames.\cite{b3} We calculate Spectral Flux by equation: Spectral Flux = $\int_{\pi_+}^{} I * \cos(\theta(n)) dw(n)$, where I is indicator function of integration that extends only over the solid angles of relevant hemisphere, $\theta(n)$ donates the angle between n and prescribed direction. \\
	\item Spectral Rolloff:\\
	The frequency below which 90\% of the magnitude distribution of the spectrum is concentrated.\cite{b3} We calculate Spectral Rolloff through equation : Spectral Rolloff = arg min$\sum_{i = 1}^{f_c} m_i \geq \sum_{i = 1}^{N}0.9 m_i$, where $f_c$ is the rolloff frequency and $m_i$ is the magnitude of the i-th frequency component of the spectrum. \\
	\item Mel Frequency Cepstral Coefficients(MFCCs):\\
	Mel Frequency Cepstral Coefficients form a cepstral representation where the frequency bands are not linear but distributed according to the mel-scale.\cite{b4} We calculate MFCC by equation: MFCC = $\sum_{k = 0}^{\frac{N}{2}}\log |s(n)| H_i(k\frac{2\pi}{N_p})$, where N is the frame length, s(n) is DFT signal, $H_i$ is the critical band filter at i-th coefficient, and $N_p$ is the number of points used in the short term DFT.\\
	\item Chroma Vector:\\
	A 12-element representation of the spectral energy where the bins represent the 12 equal-tempered pitch classes of western-type music (semitone spacing).\cite{b5} We calculate this by detecting the tone height and chroma of certain segment , and assign each element the vector it falls into.\\
	\item Chroma Deviation:\\
	The standard deviation of the 12 chroma coefficients.\cite{b5}\\
	\item Local Amplitude Minimum:\\
	The number of local minimum amplitude peaks. We extract this feature by counting the number of frames where its frequency derivative is 0.\\
\end{itemize}
First 13 features are widely used in the field of machine learning for audio classification. We utilize an existing library\cite{b6} that implements first those features. The last feature Local Amplitude Minimum is introduced by us to help classifiers distinguishing normal audio and abnormal audio, which turns out to be extremely efficient.
\section{Results \& Discussion}


We have several different kinds of models such as SVM, Label Propagation and Multiple Instance Learning in our ensemble. Some of these works well such as Label Propagation, it is able to detect 69 normal cases, but the Multiple Instance Learning is only capturing 12 normal cases out of 400, which is under calling a lot of normal examples, though there is a very high Area under ROC graph, 0.96. It seems that it is difficult to detect the normal cases. It also seems that there might be some mismatch between the testing examples and training examples. While our model are able to detect the normal examples with 87 \% in the cross validation of SVM, the actual ensemble results are far worse than our result. We can see from Table 1. that there are 27 \% difference between the two results. 


\begin{table*}[!htbp]
	\caption{SVM Cross Validation VS Actual Result}
	\begin{center}
		\begin{tabular}{|c|c|c|c|c|c|}
			\hline
			 & SVM CV & MIL CV & ensemble & with TSVM & without TSVM \\
			\hline
			Normal vs Pathological & 87.0 \%& & & 60.0 \% & 76 \%  \\
			\hline
			Volcal palsy vs rest & 89.0 \% & && 63.77 \%& 60.67 \%\\
			\hline
			Phonotrauma vs Neoplasm & 74.0 \% &&& &\\
			\hline
		\end{tabular}
		\label{tab2}
	\end{center}
\end{table*}

We have also used Transductive SVM in our previous prediction. TSVMs exploit specific iterative algorithms which gradually search a reliable separating hyperplane (in the kernel space) with a transductive process that incorporates both labeled and unlabeled samples in the training phase[tsvm].

In Table 1, we can see the different results we get after we removed the TSVM from the ensemble. The problem with TSVM is that it can only find 20 ?? normal cases, where there are about 80 normal cases in the actual test data. We do find that after removing the TSVM, we get an inscrease in the number of normal cases detected. However, the classification for pathological classes decreased by about 3 \%. We suspect that the TSVM is good for the disease classification but not for distinguishing if the example is normal or not. 
  % Zhengkai Jiang
\section{Conclusion}
We have presented an approach to classifying vocal diseases using an ensemble of different approaches. We use a custom ensemble consisting of supervised, semi-supervised and multiple-instance learning methods. We use a pipeline architecture to separate normal samples from vocal palsy, neoplasm and phonotrauma samples. Our approach was able to obtain a sensitivity of 89.4\% and specificity of 76\% on the test data with a UAR score of 60.67\%. In future work, if we obtain the labeled test set, we would like to analyze the errors of our approach to improve it further.


\begin{thebibliography}{00}
\bibitem{b30}
Kay Elemetrics. Disordered Voice Database 1.03ed, 1994.	

\bibitem{b1}
 Gouyon F., Pachet F., Delerue O. (2000) On the Use of Zero-crossing Rate for an Application of Classification of Percussive Sounds, in Proceedings of the COST G-6 Conference on Digital Audio Effects (DAFX-00 - DAFX-06), Verona, Italy, December 7–9, 2000. Accessed 26 April 2011.

\bibitem{b2}
Müller, G., Möser, M. (2012). Handbook of Engineering Acoustics. Springer. p. 7. ISBN 9783540694601.

\bibitem{b3}
Grey, J. M., Gordon, J. W., 1978. Perceptual effects of spectral modifications on musical timbres. Journal of the Acoustical Society of America 63 (5), 1493–1500, doi:10.1121/1.381843

\bibitem{b4}
Beigi, Homayoon (2011). Fundamentals of Speaker Recognition. Berlin: Springer-Verlag. ISBN 978-0-387-77591-3.

\bibitem{b5}
Müller, Meinard (2015). Fundamentals of Music Processing. Springer. doi:10.1007/978-3-319-21945-5. ISBN 978-3-319-21944-8.

\bibitem{b6}
T. Giannakopoulos. pyAudioAnalysis: An Open-Source Python Library for Audio Signal Analysis, PLoS One, 10, pp. 12, 2015.

\bibitem{b7} 
Fabian Pedregosa, Gaël Varoquaux, Alexandre Gramfort, Vincent Michel, Bertrand Thirion, Olivier Grisel, Mathieu Blondel, Peter Prettenhofer, Ron Weiss, Vincent Dubourg, Jake Vanderplas, Alexandre Passos, David Cournapeau, Matthieu Brucher, Matthieu Perrot, Édouard Duchesnay, Scikit-learn: Machine Learning in Python

\bibitem{b8} 
Lars Buitinck, Gilles Louppe, Mathieu Blondel, Fabian Pedregosa, Andreas C. Muller, Olivier Grisel, Vlad Niculae, Peter Prettenhofer, Alexandre Gramfort, Jaques Grobler, Robert Layton, Jake Vanderplas, Arnaud Joly, Brian Holt, and Ga''el Varoquaux. API design for machine learning software: experiences from the scikit-learn project.

\bibitem{b31}
P. Kukharchik, D. Martynov, I. Kheidorov and O. Kotov (2007). Vocal fold pathology detection using modified wavelet-like features and support vector machines. In Proceedings of the 15th European Signal Processing Conference, Poznan, 2007, pp. 2214-2218.

\bibitem{b32}
Majidnezhad, V. and Kheidorov, I. (2013) An ANN-based method for detecting vocal fold pathology. arXiv preprint arXiv:1302.1772.

\bibitem{b9}
Shih-Hau Fang, Yu Tsao, Min-Jing Hsiao, Ji-Ying Chen, Ying-Hui Lai, Feng-Chuan Lin, Chi-Te Wang (2018).
Detection of Pathological Voice Using Cepstrum Vectors: A Deep Learning Approach,
Journal of Voice, 2018. \url{https://doi.org/10.1016/j.jvoice.2018.02.003}
\bibitem{b10}
S. Ray \& M. Craven (2005).
Supervised versus Multiple-Instance Learning: An Empirical Comparison.
Appears in the Proceedings of the 22nd International Conference on Machine Learning, Bonn, Germany.
\bibitem{b11}
Far Eastern Memorial Hospital, FEMH. \url{https://www.femh.org.tw}
\bibitem{b12}
Cortes, Corinna; Vapnik, Vladimir N. (1995). Support-vector networks. Machine Learning. 20 (3): 273–297. doi:10.1007/BF00994018.
\bibitem{b13}
Xiaojin Zhu (2005). Learning from Labeled and Unlabeled Data with Label Propagation. PhD Thesis, Carnegie Mellon University.
\bibitem{b14}
Xiang Liao, Dezhong Yao and Chaoyi Li. Transductive SVM for reducing the training effort in BCI
\bibitem{b15}
Renyi, Alfréd (1961). "On measures of information and entropy" (PDF). Proceedings of the fourth Berkeley Symposium on Mathematics, Statistics and Probability 1960. pp. 547–561
\bibitem{b16}
“FAQ,” FEMH CHALLENGE. [Online]. Available: \url{https://femh-challenge2018.weebly.com/faq.html}. [Accessed: 12-Nov-2018].
\bibitem{b17}
F. Pedregosa, G. Varoquaux, A. Gramfort, V. Michel, B. Thirion, O. Grisel, M. Blondel, P. Prettenhofer, R. Weiss, V. Dubourg, J. Vanderplas, A. Passos, D. Cournapeau, M. Brucher, M. Perrot, E. Duchesnay (2011). Scikit-learn: Machine Learning in Python., JMLR 12, pp. 2825-2830, 2011.
\bibitem{b18}
Xiaojin Zhu, and Zoubin Ghahramani, and John Lafferty (2003). Semi-supervised Learning Using Gaussian Fields and Harmonic Functions. ICML 2003.
\bibitem{b20}
David Ferrucci, Eric Brown, Jennifer Chu-Carroll, James Fan, David Gondek, Aditya A. Kalyanpur, Adam Lally, J. William Murdock, Eric Nyberg, John Prager, Nico Schlaefer, and Chris Welty . Building Watson: An overview of the DeepQA project. AI Magazine 31, 59-79 (2010). Available: https://www.aaai.org/Magazine/Watson/watson.php
\bibitem{b21}
Yehuda Koren (2009). The BellKor solution to the Netflix grand prize". Netflix prize documentation, vol. 81, 2009.
\bibitem{b22}
R. Collobert, F. H. Sinz, J. Weston, L. Bottou (2006). Large scale transductive SVMs. J. Mach. Learn. Res., vol. 7, pp. 1687-1712, 2006.
\bibitem{b23}
Fabian Gieseke, Antti Airola, Tapio Pahikkala, and Oliver Kramer (2014). Fast and Simple Gradient-Based Optimization for Semi-Supervised Support Vector Machines. Neurocomputing (ICPRAM 2012 Special Issue) 123(10):23-32, 2014. 
\bibitem{b25}
T. Cover and J. Thomas (2006). Elements of information theory. New York: John Wiley \& Sons, 2006.
\bibitem{b26}
Rafael George Amado, Jozue Vieira Filho (2008). Pitch detection algorithms based on zero-cross rate and autocorrelation function for musical notes, IEEE, July 2008.
\bibitem{b27}
Chenn-Jung Huang,Yi-Ju Yang, Dian-Xiu Yang,You-Jia Chen (2008). Frog classification using machine learning techniques, Expert Systems with Applications Volume 36, Issue 2, Part 2, March 2009, Pages 3737-3743, March 2008
\end{thebibliography}
\end{document}
