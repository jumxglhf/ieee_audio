\section{Results \& Discussion}


We have several different kinds of models including TSVM, SVM, Label Propagation and MILR in our ensemble. 

Some of these works well such as Label Propagation, it is able to detect 69 normal cases, but the MILR is only capturing far fewer normal cases. However, there is a very high Area under ROC graph for MILR, 0.96. It seems that it is difficult to detect the normal cases. For TSVM, it is overcalling a lot of normal cases with the default 50\% confidence threshold, which have the opposite the problem compared with MILR. So we think that there might be some mismatches between the testing examples and training examples, probably there are some features that are not included in the training examples but appeared in the testing files. It is very obvious to see that the UAR for cross validation of SVM is 76.40\% and the UAR for test cases is only 64.77\%. So we have to change the confidence level in order to get a closer result.

We are able to detect the normal examples with 79.5\% in the cross validation for SVM, [number] \% for the MILR, but the actual result for the test cases is 60.0\%. We can see from Table 1. that there are 20\% difference between the two results. Since the other two methods are Label Propagation and TSVM, which are not able to do the cross validation, we can only analyze the SVM and MILR.


\begin{table*}[!htbp]
	\caption{CROSS VALIDATION VS ACTUAL RESULT}
	\begin{center}
		\begin{tabular}{|c|c|c|c|c|c|c|}
			\hline
			 & SVM CV & SVM-RBF & MILR CV & ensemble & with TSVM & without TSVM \\
			\hline
			Sensitivity$^{\mathrm{a}}$  & 88.8 \%& && & 82.2 \% & 89.4 \% \\
			\hline
			Specificity & 79.5 \% &&&& 60.0 \% & 76.0 \% \\
			\hline
			Volcal palsy recall & 81.0 \% && && N/A & N/A \\
			\hline
			Phonotrauma recall & 61.0 \% &70.0 \%&&& N/A& N/A\\
			\hline
			Neoplasm recall & 81.1 \% &67.5 \%&&& N/A& N/A \\
			\hline
			UAR & 76.40 \% & && & 64.77 \% & 60.67 \%\\
			\hline
			\multicolumn{4}{l}{$^{\mathrm{a}}$ Number of correctly predicted pathological / total pathological}
		\end{tabular}
		\label{tab2}
	\end{center}
\end{table*}

Then, we tried another ensemble without the TSVM. We suspect that the there are problems caused by the over calling of normal by it. We changed the TSVM with the SVM-RBF. Then, we see n increase in both the sensitivity and specificity but the UAR decreased. From Table 1, we can see that there are 4\% decrease in UAR score. We think that the SVM-RBF works better for distinguishing normal examples from pathological examples but the TSVM might work better for classification the different kinds of diseases. There might be some features that TSVM could capture but the SVM-RBF cannot. One possible reason might be that the two diseases Phonotrauma and Neoplasm are very close and hard to distinguish for SVM-RBF, we can see from Table 1, that the SVM-RBF only have around 70 \% of recall for both classes.
 
